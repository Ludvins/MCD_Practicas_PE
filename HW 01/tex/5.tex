
\textbf{Exercise 5. }\emph{Consider the process \( X(t) = Z\sqrt{t} \) for \( t \geq 0 \) with the same value of \( Z \) for all \( t \).}
\begin{enumerate}
  \item[\textit{(i)}] \emph{Show that the distribution of the process at time \( t \) is the same as that of a Wiener process\footnote{The notation $\mathcal N(\mu, \sigma)$ used in this document assumes that $\sigma$ is the \textit{standard deviation} of the distribution, and not the variance.}: \( X(t) \sim \mathcal{N}(0, \sqrt{t}) \)}.
  \item[\textit{(ii)}] \emph{What is the mathematical property that allows us to prove that this process is not Brownian?}
\end{enumerate}

\emph{Solution.} Let \( Z \sim \mathcal{N}(0, 1) \). We know that the family of normal distributions is closed under linear transformations, and more specifically, that multiplying a zero-mean normal variable by a constant $a>0$ yields another zero-mean normally-distributed variable for which the standard deviation is the old one times $a$. So it is immediate to see that
\[
X(t) = Z \sqrt{t} \sim \sqrt{t}\mathcal{N}(0, 1) = \mathcal{N}(0, \sqrt t).\]

To prove that this process is not Brownian, the key property that we have to use is that of \textit{independence} (or lack thereof). Indeed, since the variable $Z$ is the same for all $t$, given $t_2>t_1\geq s_2> s_1\geq 0$, we have
\[
X(t_2) - X(t_1) = Z(\sqrt{t_2} - \sqrt{t_1}) \quad \text{and} \quad X(s_2) - X(s_1) = Z(\sqrt{s_2} - \sqrt{s_1}),
\]
which are clearly not independent. Since independent increments are an essential property of Brownian processes, $X(t)$ cannot be Brownian.\\
