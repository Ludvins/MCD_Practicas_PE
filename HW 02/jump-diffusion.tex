%%% Local Variables:
%%% LaTeX-command: "pdflatex --shell-escape"
%%% End:

\documentclass[11pt]{article}
\usepackage[utf8]{inputenc}
\usepackage[T1]{fontenc}
\usepackage{graphicx}
\usepackage{grffile}
\usepackage{longtable}
\usepackage{wrapfig}
\usepackage{rotating}
\usepackage[normalem]{ulem}
\usepackage{amsmath}
\usepackage{textcomp}
\usepackage{amssymb}
\usepackage{capt-of}
\usepackage{hyperref}
\hypersetup{colorlinks=true, linkcolor=magenta}
\setlength{\parindent}{0in}
\usepackage[margin=0.8in]{geometry}
\usepackage[english]{babel}
\usepackage{mathtools}
\usepackage{fancyhdr}
\usepackage{sectsty}
\usepackage{engord}
\usepackage{parskip}
\usepackage{minted}
\usepackage{cite}
\usepackage{graphicx}
\usepackage{setspace}
\usepackage[compact]{titlesec}
\usepackage[center]{caption}
\usepackage{placeins}
\usepackage{color}
\usepackage{amsmath}
\usepackage{bm}
\usepackage{minted}
\usepackage{subfigure}
\usepackage{pdfpages}
% \titlespacing*{\subsection}{0pt}{5.5ex}{3.3ex}
% \titlespacing*{\section}{0pt}{5.5ex}{1ex}
\author{Luis Antonio Ortega Andrés\\Antonio Coín Castro}
\date{\today}
\title{Continuous-time stochastic processes\\\medskip
\large Homework 1}
\hypersetup{
 pdfauthor={Luis Antonio Ortega Andrés},
 pdftitle={Ejercicios de clase},
 pdfkeywords={},
 pdfsubject={},
 pdflang={Spanish}}

% \usemintedstyle{bw}

\begin{document}

\textbf{Jump-adapted Euler scheme for jump-diffusion SDEs}

We wish to simulate the following SDE with jumps:
\[
X(t_0)=x_0, \quad
dX(t)=a(t,X(t))\,dt + b(t, X(t))\,dW(t)+c(t, X(t^-))\, dJ(t).
\]
Let $\{t^*_1 < t_2^* < \dots < t^*_{N_1}=T\}$ be a regular grid of $N_1$ points, and let $\{\tau_1 < \tau_2 < \dots < \tau_{N_2}\}$ be the set of jump times simulated in $[t_0,t_0+T]$, each with intensity $Y_j \sim f_{Y}(y)$. We consider the following jump-adapted (non-regular) grid,
\[
\{t_1,\dots, t_{N}\}= \{t^*_1,\dots t^*_{N_1}\} \cup \{\tau_1,\dots,\tau_{N_2}\},
\]

and we define $\Delta T_n = t_{n+1}-t_{n}$ for all $n$. For convenience, let $\xi_n$ denote the approximation of $X(t)$ at time $t_n$. For a single simulation, we proceed as follows:

\begin{enumerate}
  \item Set $\xi_0=x_0$.
  \item For a generic approximation $\xi_{n+1}$, we perform the following calculations:
  \begin{itemize}
    \item Set $\xi_{n+1}^-=\xi_n + a(t_n, \xi_n)\Delta T_n+b(t_n, Y_n)\sqrt{\Delta T_n}Z_n$, where $Z_n \sim \mathcal N(0,1)$ iid.
    \item Take the possible jump into account by setting the definitive approximation as  \[\xi_{n+1}=\xi_{n+1}^- + c(t_{n+1}, \xi_{n+1}^-)\Delta J_n,\]
    where $\Delta J_n$ is defined as:
    \[
    \Delta J_n =\begin{cases}
      Y_j& \text{if } $\exists j: t_{n+1}=\tau_j,$\\
      0& \text{otherwise}.
    \end{cases}
    \]
  \end{itemize}
\end{enumerate}

Note that in each step there can be at most one jump (since the jump times are included in the grid). Now there are two options for the return values:

\begin{itemize}
  \item \textbf{Option 1.} Return the complete set of approximations $\{\xi_0,\dots, \xi_{N}\}$ at non-regular time steps.
  \item \textbf{Option 2.} Return only those approximations corresponding to the original regular grid of $N_1$ points.
\end{itemize}

Option 2 seems to be the most reasonable choice, since in Option 1 we would get different grid sizes for each simulation.
\end{document}
